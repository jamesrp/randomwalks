\documentclass[12pt]{amsart}
\usepackage{amssymb,amsmath}
\usepackage{amsthm}
\theoremstyle{plain}
\newtheorem{theorem}{Theorem}[section]
\newtheorem{lemma}[theorem]{Lemma}
\newtheorem{corollary}[theorem]{Corollary}
\newtheorem{proposition}[theorem]{Proposition}

\theoremstyle{definition}
\newtheorem{definition}[theorem]{Definition}
\newtheorem{example}[theorem]{Example}

\theoremstyle{remark}
\newtheorem{remark}[theorem]{Remark}

\title{Double cover time}

\begin{document}
\maketitle
\section{Gambler's ruin lemmas}
We begin with some easy facts about gambler's ruin - i.e., random walk on a line graph.
In what follows, we consider a random walk on ${0, ..., n}$ with nearest neighbor adjacency.
We start at $i$, take unbiased random steps left or right, and stop when we reach either $0$ or $n$.

The first two results are commonly proved in introductory textbooks.
One conditions on the first step and solves the resulting system of equations.

\begin{lemma}
$P(\textup{reach 0 first}) = \frac{n-i}{n}$.
$P(\textup{reach $n$ first}) = \frac{i}{n}$.
\end{lemma}

Let $T$ be number of steps taken by the walk until it stops.

\begin{lemma}
$E(T) = i(n-i)$.
\end{lemma}

The next statement is elementary, with proof similar to the first two, but is not in most textbooks.
It must be known, but I could not find a reference, so I will prove it here.

\begin{lemma}
$E(T|\textup{reach 0 first}) = \frac{1}{3}i(2n-i)$.
$E(T|\textup{reach $n$ first}) = \frac{1}{3}(n-i)(n+i)$.
\end{lemma}
\begin{proof}
The second statement follows from the first by the change of variables $i \mapsto n-i$.
We will prove the first.
Let $e_i = E(T|\textup{reach 0 first})$ if we start at $i$.
Then $e_0 = 0$ and $e_{n-1} = 1 + e_{n-2}$.
For $1 \le i \le n-2$, we can condition on the first step $s$.
\begin{equation}
e_i = 1 + e_{i-1}P(s=-1|\textup{reach 0 first}) + e_{i+1}P(s=1|\textup{reach 0 first})
\end{equation}
To compute these probabilities is a simple application of conditional probability formula and Lemma 1.1.
For example:
\begin{align*}
P(s=-1|\textup{reach 0 first}) &= P(s=-1,\textup{reach 0 first}) / P(\textup{reach 0 first}) \\
&= \left(\frac{1}{2} \cdot \frac{n-i+1}{n}\right)/\left(\frac{n-i}{n}\right) \\
&= \frac{n-i+1}{2(n-i)}
\end{align*}
Substituting these into (1), we can observe that the resulting system of equations admits at most one solution.
Therefore, we can simply check that $e_i = \frac{1}{3}i(2n-i)$ solves the system.
\end{proof}

\end{document}
