\documentclass[12pt]{amsart}
\usepackage{amssymb,amsmath}
\usepackage{amsthm}
\theoremstyle{plain}
\newtheorem{theorem}{Theorem}[section]
\newtheorem{lemma}[theorem]{Lemma}
\newtheorem{corollary}[theorem]{Corollary}
\newtheorem{proposition}[theorem]{Proposition}

\theoremstyle{definition}
\newtheorem{definition}[theorem]{Definition}
\newtheorem{example}[theorem]{Example}

\theoremstyle{remark}
\newtheorem{remark}[theorem]{Remark}

\title{Double cover time}

\begin{document}
\maketitle
\section{Gambler's ruin lemmas}
We begin with some easy facts about gambler's ruin - i.e., random walk on a line
graph.
In what follows, we consider a random walk on ${0, ..., n}$ with nearest neighbor
adjacency.
We start at $i$, take unbiased random steps left or right, and stop when we
reach either $0$ or $n$.

The first two results are commonly proved in introductory textbooks.
One conditions on the first step and solves the resulting system of equations.

\begin{lemma}
$P(\textup{reach 0 first}) = \frac{n-i}{n}$.
$P(\textup{reach $n$ first}) = \frac{i}{n}$.
\end{lemma}

Let $T$ be number of steps taken by the walk until it stops.

\begin{lemma}
$E(T) = i(n-i)$.
\end{lemma}

The next statement is elementary, with proof similar to the first two, but is
not in most textbooks. 

\begin{lemma}
$E(T|\textup{reach 0 first}) = \frac{1}{3}i(2n-i)$.
$E(T|\textup{reach $n$ first}) = \frac{1}{3}(n-i)(n+i)$.
\end{lemma}
\begin{proof}
TODO
\end{proof}

\end{document}
